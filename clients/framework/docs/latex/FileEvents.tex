\section{Reacting to file events.}\label{FileEvents}


The class {\bf CFile\-Event} {\rm (p.\,\pageref{classCFileEvent})} allows an application to react asynchronously to events on a file. To use a {\bf CFile\-Event} {\rm (p.\,\pageref{classCFileEvent})} you will need to:\begin{CompactItemize}
\item 
Decide which of the following conditions you need to supply  application code for:\begin{enumerate}
\item 
Readbility (may be significant on e.g. \item 
Writability (may be significant e.g. pipes).\item 
Exceptional conditions (e.g. pipe broken).\end{enumerate}
\item 
Write a subclass of CFile event which supplies the appropriate application specific functionality for any or all of the functions:\begin{enumerate}
\item 
{\bf On\-Readable} {\rm (p.\,\pageref{classCFileEvent_a15})} which is called when the file descriptor becomes readable.\item 
{\bf On\-Writable} {\rm (p.\,\pageref{classCFileEvent_a16})} which  is called when the file descriptor becomes writable.\item 
{\bf On\-Exception} {\rm (p.\,\pageref{classCFileEvent_a17})} which is called when the file descriptor has an exceptional condition pending.\end{enumerate}
\item 
During application execution create and enable an instance of this class.\end{CompactItemize}
In the example below, a file event is created to monitor stdin. When input is available on stdin, the first word of the line is echoed back on stdout.



\footnotesize\begin{verbatim}#include <iostream.h>
#include <stdio.h>
#include <spectrodaq.h>
#include <SpectroFramework.h>

class Echo : public CFileEvent
{
public:
  Echo(int fd, const char* pName);
  virtual void OnReadable(istream& rin);
};

Echo::Echo(int fd, const char* pName):
  CFileEvent(fd, pName)
{
  AppendClassInfo();
}

void
Echo::OnReadable(istream& rin)
{
  CFileEvent::OnReadable(rin);
  string word;
  rin >> word;
  cout << word << endl;
}

class MyApp : public DAQROCNode
{
protected:
  virtual int operator()(int argc, char** argv);
};

int
MyApp::operator()(int argc, char** argv)
{
  Echo echo(fileno(stdin), "EchoProcessor");

  echo.Enable();
  DAQThreadId id = echo.getThreadId();

  Join(id);                     // Wait for echo to exit.
}


MyApp theapp;
\end{verbatim}\normalsize 


