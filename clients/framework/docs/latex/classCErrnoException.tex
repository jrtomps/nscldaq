\section{CErrno\-Exception  Class Reference}
\label{classCErrnoException}\index{CErrnoException@{CErrno\-Exception}}
{\tt \#include $<$CErrno\-Exception.h$>$}

Inheritance diagram for CErrno\-Exception::\begin{figure}[H]
\begin{center}
\leavevmode
\includegraphics[height=3cm]{classCErrnoException}
\end{center}
\end{figure}
\subsection*{Public Methods}
\begin{CompactItemize}
\item 
{\bf CErrno\-Exception} (const char $\ast$psz\-Action)
\item 
{\bf CErrno\-Exception} (const std::string \&rs\-Action)
\item 
{\bf $\sim$CErrno\-Exception} ()
\item 
{\bf CErrno\-Exception} (const CErrno\-Exception \&a\-CErrno\-Exception)
\item 
CErrno\-Exception \& {\bf operator=} (const CErrno\-Exception \&a\-CErrno\-Exception)
\item 
int {\bf operator==} (const CErrno\-Exception \&a\-CErrno\-Exception)
\item 
{\bf Int\_\-t} {\bf get\-Errno} () const
\item 
virtual const char $\ast$ {\bf Reason\-Text} () const
\item 
virtual {\bf Int\_\-t} {\bf Reason\-Code} () const
\end{CompactItemize}
\subsection*{Protected Methods}
\begin{CompactItemize}
\item 
void {\bf set\-Errno} ({\bf Int\_\-t} am\_\-n\-Errno)
\end{CompactItemize}
\subsection*{Private Attributes}
\begin{CompactItemize}
\item 
{\bf Int\_\-t} {\bf m\_\-n\-Errno}
\end{CompactItemize}


\subsection{Constructor \& Destructor Documentation}
\index{CErrnoException@{CErrno\-Exception}!CErrnoException@{CErrnoException}}
\index{CErrnoException@{CErrnoException}!CErrnoException@{CErrno\-Exception}}
\subsubsection{\setlength{\rightskip}{0pt plus 5cm}CErrno\-Exception::CErrno\-Exception (const char $\ast$ {\em psz\-Action})\hspace{0.3cm}{\tt  [inline]}}\label{classCErrnoException_a0}




Definition at line 325 of file CErrno\-Exception.h.

References m\_\-n\-Errno.\index{CErrnoException@{CErrno\-Exception}!CErrnoException@{CErrnoException}}
\index{CErrnoException@{CErrnoException}!CErrnoException@{CErrno\-Exception}}
\subsubsection{\setlength{\rightskip}{0pt plus 5cm}CErrno\-Exception::CErrno\-Exception (const std::string \& {\em rs\-Action})\hspace{0.3cm}{\tt  [inline]}}\label{classCErrnoException_a1}




Definition at line 328 of file CErrno\-Exception.h.

References m\_\-n\-Errno.\index{CErrnoException@{CErrno\-Exception}!~CErrnoException@{$\sim$CErrnoException}}
\index{~CErrnoException@{$\sim$CErrnoException}!CErrnoException@{CErrno\-Exception}}
\subsubsection{\setlength{\rightskip}{0pt plus 5cm}CErrno\-Exception::$\sim$CErrno\-Exception ()\hspace{0.3cm}{\tt  [inline]}}\label{classCErrnoException_a2}




Definition at line 331 of file CErrno\-Exception.h.\index{CErrnoException@{CErrno\-Exception}!CErrnoException@{CErrnoException}}
\index{CErrnoException@{CErrnoException}!CErrnoException@{CErrno\-Exception}}
\subsubsection{\setlength{\rightskip}{0pt plus 5cm}CErrno\-Exception::CErrno\-Exception (const CErrno\-Exception \& {\em a\-CErrno\-Exception})\hspace{0.3cm}{\tt  [inline]}}\label{classCErrnoException_a3}




Definition at line 335 of file CErrno\-Exception.h.

References m\_\-n\-Errno.

\subsection{Member Function Documentation}
\index{CErrnoException@{CErrno\-Exception}!getErrno@{getErrno}}
\index{getErrno@{getErrno}!CErrnoException@{CErrno\-Exception}}
\subsubsection{\setlength{\rightskip}{0pt plus 5cm}{\bf Int\_\-t} CErrno\-Exception::get\-Errno () const\hspace{0.3cm}{\tt  [inline]}}\label{classCErrnoException_a6}




Definition at line 361 of file CErrno\-Exception.h.

References Int\_\-t, and m\_\-n\-Errno.\index{CErrnoException@{CErrno\-Exception}!operator=@{operator=}}
\index{operator=@{operator=}!CErrnoException@{CErrno\-Exception}}
\subsubsection{\setlength{\rightskip}{0pt plus 5cm}CErrno\-Exception\& CErrno\-Exception::operator= (const CErrno\-Exception \& {\em a\-CErrno\-Exception})\hspace{0.3cm}{\tt  [inline]}}\label{classCErrnoException_a4}




Definition at line 342 of file CErrno\-Exception.h.

References m\_\-n\-Errno, and CException::operator=().

Referenced by CTCPNo\-Such\-Service::operator=(), CTCPConnection\-Lost::operator=(), and CTCPConnection\-Failed::operator=().\index{CErrnoException@{CErrno\-Exception}!operator==@{operator==}}
\index{operator==@{operator==}!CErrnoException@{CErrno\-Exception}}
\subsubsection{\setlength{\rightskip}{0pt plus 5cm}int CErrno\-Exception::operator== (const CErrno\-Exception \& {\em a\-CErrno\-Exception})\hspace{0.3cm}{\tt  [inline]}}\label{classCErrnoException_a5}




Definition at line 352 of file CErrno\-Exception.h.

References m\_\-n\-Errno, and CException::operator==().

Referenced by CTCPNo\-Such\-Service::operator==(), CTCPConnection\-Lost::operator==(), and CTCPConnection\-Failed::operator==().\index{CErrnoException@{CErrno\-Exception}!ReasonCode@{ReasonCode}}
\index{ReasonCode@{ReasonCode}!CErrnoException@{CErrno\-Exception}}
\subsubsection{\setlength{\rightskip}{0pt plus 5cm}{\bf Int\_\-t} CErrno\-Exception::Reason\-Code () const\hspace{0.3cm}{\tt  [virtual]}}\label{classCErrnoException_a8}


Return an integer valued code which indicates the reason for the exception. This value is exception type dependent. for CErrno\-Exception, the value is the value of errno at the time the exception was instantiated. 

Reimplemented from {\bf CException} {\rm (p.\,\pageref{classCException_a9})}.

Definition at line 342 of file CErrno\-Exception.cpp.

References m\_\-n\-Errno.\index{CErrnoException@{CErrno\-Exception}!ReasonText@{ReasonText}}
\index{ReasonText@{ReasonText}!CErrnoException@{CErrno\-Exception}}
\subsubsection{\setlength{\rightskip}{0pt plus 5cm}const char $\ast$ CErrno\-Exception::Reason\-Text () const\hspace{0.3cm}{\tt  [virtual]}}\label{classCErrnoException_a7}


Return the textual reason for the error. This is done by asking for the strerror of m\_\-n\-Errno. Note therefore that the reason text returned relates to the value of errno at the time the exception was thrown, not the current time. 

Reimplemented from {\bf CException} {\rm (p.\,\pageref{classCException_a8})}.

Reimplemented in {\bf CTCPConnection\-Failed} {\rm (p.\,\pageref{classCTCPConnectionFailed_a7})}, {\bf CTCPConnection\-Lost} {\rm (p.\,\pageref{classCTCPConnectionLost_a7})}, and {\bf CTCPNo\-Such\-Service} {\rm (p.\,\pageref{classCTCPNoSuchService_a6})}.

Definition at line 323 of file CErrno\-Exception.cpp.

References m\_\-n\-Errno.

Referenced by CTCPNo\-Such\-Service::Reason\-Text(), CTCPConnection\-Lost::Reason\-Text(), and CTCPConnection\-Failed::Reason\-Text().\index{CErrnoException@{CErrno\-Exception}!setErrno@{setErrno}}
\index{setErrno@{setErrno}!CErrnoException@{CErrno\-Exception}}
\subsubsection{\setlength{\rightskip}{0pt plus 5cm}void CErrno\-Exception::set\-Errno ({\bf Int\_\-t} {\em am\_\-n\-Errno})\hspace{0.3cm}{\tt  [inline, protected]}}\label{classCErrnoException_b0}




Definition at line 367 of file CErrno\-Exception.h.

References Int\_\-t, and m\_\-n\-Errno.

\subsection{Member Data Documentation}
\index{CErrnoException@{CErrno\-Exception}!m_nErrno@{m\_\-nErrno}}
\index{m_nErrno@{m\_\-nErrno}!CErrnoException@{CErrno\-Exception}}
\subsubsection{\setlength{\rightskip}{0pt plus 5cm}{\bf Int\_\-t} CErrno\-Exception::m\_\-n\-Errno\hspace{0.3cm}{\tt  [private]}}\label{classCErrnoException_o0}




Definition at line 319 of file CErrno\-Exception.h.

Referenced by CErrno\-Exception(), get\-Errno(), operator=(), operator==(), Reason\-Code(), Reason\-Text(), and set\-Errno().

The documentation for this class was generated from the following files:\begin{CompactItemize}
\item 
{\bf CErrno\-Exception.h}\item 
{\bf CErrno\-Exception.cpp}\end{CompactItemize}
